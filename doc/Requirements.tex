
\chapter{Requirements}
\label{chapt:requirements}

In this section we detail the requirements of the Random Thoughts generator. These requirements
are used to drive the architecture and design of the system.

Random Thoughts consists of two parts: a hardware generator that attaches to the computer using
a standard USB version 2.0 connection, and a command interface for communicating with the
generator hardware. The generator's direct I/O facilities shall consist of three LEDs as
follows.

\begin{enumerate}
\item \textit{Green}: The generator is functioning correctly and has available entropy.

\item \textit{Yellow}: The generator is functioning correctly but has limited or no entropy.
  This condition should correct itself automatically after the generator has accumulated entropy
  from its random source. Using the generator at this time is permitted but the quality of the
  random numbers produced may be poor.

\item \textit{Red}: The generator is not functioning properly. It should not be used in this
  condition. Automatic correction of this condition is not expected.
\end{enumerate}

Setting up the generator shall only require connecting it to the computer. The generator draws
all its power from the USB connection. No batteries or other adjustments are required.

In addition the generator has a command language that details how the driver can interact with
the hardware. This command language shall be usable by third parties for building drivers on
systems not otherwise supported.

At the level of the command language only two requests need be supported. The first is a
non-blocking request for a specific number of bytes of random data that must be ``immediately''
delivered, and a blocking request for a specific number of bytes of random data that must be
delivered eventually. Immediate delivery means that the data is taken from a buffer in the
generator. The request does not have to wait for the generator to produce all of the requested
data. If the generator does not have all of the requested data on hand, it should return what it
does have along with an appropriate count (which might be zero).

When a blocking request is issued, the generator attempts to return all of the data requested,
perhaps after a delay to generate that data. The data does not have to be returned all at once.
The generator may return the data in several bursts depending on the size of the request and the
size of the generator's on-board buffer.

Note that the detailed format of the messages used in the command language is not specified
here. In addition the command language must allow the generator hardware to signal an error
condition to the calling software. This error condition is intended to indicate a hardware
failure or bad statistics but the specific kinds of errors that are to be detected are not
specified here. In particular the generator may or may not be capable of doing a statistical
analysis of the numbers it produces. However, error codes for statistical problems should at
least be reserved to streamline the development of future generators.

\section{Core Requirements}
\label{sec:core-requirements}

The following requirements apply:

\begin{itemize}
\item \textbf{Core.GenSPARK}. The code on the generator hardware itself shall be in SPARK 2014
  and proved free of the possibility of runtime error.
\item \textbf{Core.Ravenscar}. The code on the generator hardware itself shall be written to
  conform with the restrictions of the Ravenscar profile.\footnote{This is likely implied by
    \textbf{Core.GenSPARK} anyway}.
\item \textbf{Core.TestsAda}. The statistical testing programs that run on the host shall be
  written in Ada.
\item \textbf{Core.Monitor}. The generator shall have a monitoring function that alerts the user
  if the statistics of the generated random numbers is poor.
\end{itemize}

The details of \textbf{Core.Monitor} are left unspecified. However, the monitoring function must
have access to enough generator data to compute reasonable conclusions about the quality of the
generator's output. It is unspecified if the statistical monitoring function is implemented in
the generator hardware or in the driver or someplace else.

The statistical monitoring discussed here is distinct from the entropy monitoring indicated by
the LEDs on the generator hardware. It is possible for the generator to believe it has
sufficient entropy to produce random numbers and yet produce such numbers with low quality. The
statistical monitoring function serves as a software check against biases that might be inherent
in the generation process.

\section{Non-Functional Requirements}
\label{sec:non-functional}

In this section we describe the non-functional requirements of Random Thoughts. The following
requirements apply:

\begin{itemize}

\item \textbf{NonFun.Review}. All software shall be subjected to a careful security analysis
  using both human review and automated tools where appropriate. The precise tools used are not
  specified here but they should be selected based availability and the current state of the
  art.

\item \textbf{NonFun.Platform}. Random Thoughts shall support both 64 bit Ubuntu Linux 20.04
  LTS. The low level interface shall be sufficiently well documented so that third parties could
  reasonably create drivers or interface libraries for other systems.

\item \textbf{NonFun.Deployment}. Software installers, with appropriate digital signatures,
  shall be provided for all supported systems in a manner that is standard for each system.
  All software shall be provided in source form with full documentation to enable (and
  encourage) review.

\item \textbf{NonFun.Performance}. The hardware shall be capable of producing random 256 bit
  session keys at the rate of one key per minute or greater. The system shall cache random data
  in such a way that it can satisfy a request for up to eight 256 bit random keys
  ``immediately'' provided there is sufficient lead time.

  Memory consumption and CPU utilization of the system shall be ``negligible.'' No network
  bandwidth shall be used by the system when it is in normal operation. The system shall consume
  one USB slave position on the host.

\item \textbf{NonFun.Security}. The standard assumption shall be made that an attacker has full
  knowledge of all algorithms, software, and hardware that is used. It shall also be assumed
  that the attacker has no physical access to the hardware during its operation (specifically
  during the production of the n bits mentioned below), nor has any administrative access on the
  machine to which Random Thoughts is connected.

  The security characteristic of Random Thoughts is as follows: Given knowledge of $n$ bits of
  random output, where $n$ is in the range $0 \le n \le 2^{32+3}$, an attacker can predict the
  next bit with probability of no more than 0.5001.

\item \textbf{NonFun.Scale}. The system shall be able to service multiple users of the host
  machine simultaneously, but note that it's overall performance as detailed above need not
  affected by the number of users. For example, two users may see an average rate of random
  number production of one 256 bit key every two minutes, etc. The generator only needs to
  service a single computer at a time.

\item \textbf{NonFun.Documentation}. The following documents shall be made available to users
  for review:
  \begin{itemize}
  \item \textit{Design Documentation}. There shall be a document containing a detailed design of
    Random Thoughts describing how the various components of the system work. It shall include
    block diagrams, schematics, or code snippets, as appropriate.
  \item \textit{Test Documentation}. There shall be a document describing how Random Thoughts
    and its associated software was tested, along with mathematical justification of the
    techniques used.
  \item \textit{User Documentation}. There shall be a document describing how to install,
    verify, and use Random Thoughts. In addition this document shall describe the low level
    command language understood by the system so third parties can build their own interfaces if
    desired.
  \item \textit{Web Site}. There shall be a web site that describes Random Thoughts and that
    provides software updates and other up-to-the-minute information about the system.
  \end{itemize}

\item \textbf{NonFun.Formats}. The communication protocol between the hardware and software
  components of the system shall be proprietary but also documented.

\item \textbf{NonFun.Internationalization}. All user interface elements and documentation shall
  be in English and nationalized for the United States. As an extension the system may make use
  of localization information provided by the host operating system.

\item \textbf{NonFun.Environment}. Random Thoughts shall be able to operate in typical office
  conditions. No special environmental hardening is required. However, the addition of shielding
  from electromagnetic snooping is a likely extension. The hardware component of the system
  shall be no larger than 8'' x 4'' x 4'' and ideally as small as feasible.

\end{itemize}