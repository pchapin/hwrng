
\chapter{Architecture}
\label{chapt:architecture}

In this section we describe the high level architecture of Random Thoughts. This includes both
the architecture of the system and the architecture of the development environment.

\section{System Architecture}

The system architecture describes the structure of the deployed system. There are three aspects
of the system to consider:

\begin{enumerate}
\item The hardware controller and noise generating hardware used as the source of randomness.
\item The command language used between the host computer and the hardware controller.
\item The driver and library software on the host computer.
\end{enumerate}

Each of these aspects are described in separate subsections below.

\subsection{Hardware Architecture}

The hardware controller consists of a low cost, single board microcontroller based on an ARM
architecture processor. Specifically, we use the STM32F4DISCOVERY with the STM32F407VG
microcontroller (32~bit ARM Cortex M4). In normal operation the controller is connected to a
host computer as a USB slave via a USB 2.0 (or higher) interface. A virtual serial port is
created on the USB interface so the controller appears as a serially connected device from the
point of view of the host computer.

Also connected to the controller, via it's general purpose I/O interface, is a supplementary
board containing a white noise generator together with an appropriate analog to digital
converter (ADC) to convert the generated noise into a stream of digital values. The word size of
the ADC is 8~bits; if more bits are available, only the lower 8~bits are used.

A software task on the controller reads the ADC ten times per second and appends the 8 bit value
read to a buffer of random data 2~KiB in size called the ``entropy pool.'' The part of the pool
that is filled with data is called the ``active pool.'' In normal operation, the pool is
completely filled, and the active pool is the same as the whole pool. Initially, while the pool
is filling (or later after some data has been extracted from it), the yellow LED on the
controller is lite. When the pool is completely filled the green LED on the controller lights
and the yellow LED turns off. The red LED on the controller is used to indicate an error
condition.

Once the pool is filled, whenever a new value is read from the ADC, the following steps are
taken:

\begin{enumerate}
\item The entire pool is hashed using SHA-256 and the resulting hash value is appended to the
  pool, causing the oldest 32~bytes to be discarded.
\item The new value from the ADC is appended to the pool, causing the oldest byte to be
  discarded.
\end{enumerate}

Thus 33~bytes are discarded from the pool for each new byte added. However, those bytes
participate in the hash that gets appended (and that hash is further hashed again, etc.).

A second task on the controller interacts with the USB interface, receiving commands from the
host computer, interpreting those commands, and acting on them. This ``interface task'' prepares
random data from the entropy pool in a separate buffer area with a size equal to the amount of
random data requested (which might be smaller or larger than the size of the active pool). The
task takes data from the pool, oldest data first, thereby reducing the active pool size, waiting
for more data to appear in the pool if necessary to completely fill the result buffer. Note that
when the pool is less than completely filled, the hashing process described above is temporarily
suspended until the pool is filled.

\subsection{Command Language}

This subsection describes the command language used between the host computer and the hardware
controller. These commands and their corresponding responses flow over a serial connection and,
for convenience, are entirely ASCII text. The generator hardware never sends an unsolicited
response. Furthermore, every command sent by the PC elicits a response of some kind (even if
only a success code). This behavior facilities testing the generator hardware using a standard
terminal emulation program.

Every command and every response is terminated by a CR/LF pair. Lines will never exceed 80
characters in length. If a command line is longer than 80 characters, the extra characters are
ignored. Unless otherwise mentioned below, commands and responses are only one line. Unless
otherwise mentioned below, responses are returned in a "timely" manner. Commands are case
insensitive.

\begin{verbatim}
Command: CHECK
Purpose:
  Verify proper operation of the generator hardware

Possile responses:
   "!OK"
   "!OK: STATISTICS UNKNOWN"
   "!HARDWARE FAILURE: [message]"
   "!BAD STATISTICS"
   "!SOFTWARE FAILURE: [message]"
   "!UNKNOWN ERROR."
\end{verbatim}

Note that support for hardware evaluation of statistics is not required. The generator hardware
should probably provide a way to tell the calling software if it does statistical analysis or
not. How that might be done is currently unspecified.

\begin{verbatim}
Command: GET IMMEDIATE n
Purpose:
  Obtain n bits of random data "immediately" (without blocking)

Possible responses:
  A string of '0' and '1' characters of size n bits
  One of the same responses to the CHECK command
  "!INSUFFICIENT DATA: n"
\end{verbatim}

Errors are indicated as for the CHECK command. The !INSUFFICIENT DATA response is sent if there
is not enough entropy to satisfy the request immediately. The value of $n$ returned by
!INSUFFICIENT DATA is the number of bits currently in the buffer. A GET IMMEDIATE with that
value of $n$ or less will definitely succeed. Use GET IMMEDIATE 0 to elicit an insufficient data
response regardless of the amount of data available, and to thus learn how much data is
available.

If <n> is greater than 80, the data will be delivered in multiple lines. The precise number of
lines is unspecified, but no line will be blank. The value of $n$ in the GET IMMEDIATE command
must be greater than or equal to zero.

\begin{verbatim}
Command: GET n
Purpose:
  Obtain n bits of random data, possibly blocking

Possible responses:
  A string of '0' and '1' characters of size n bits
  One of the same responses to the CHECK command
\end{verbatim}

The returned data might be possibly split over several lines as described above, until a total
of $n$ bits are returned. If all the bits are not yet available, the generator hardware will
wait until they are, either returning the bits as they are produced (meaning in spurts) or in a
single block once they are all produced. The generator hardware will accept no other commands
while it is generating data. However, the unit may output an error response (as defined for the
CHECK command) on a line by itself at any time during the random output.

\subsection{Driver/Library Architecture}

There are two parts to the software that runs on the computer host. The first part is a kernel
level driver that presents a device file such that when the device file is read, random binary
data is returned following the usual semantics for reading devices. The driver should also
provide a ``raw'' mode that allows applications to write commands directly in the command
language described above and read the corresponding responses. This allows applications to be
built that interact directly with the controller hardware for special purposes.

The second software component is a library that uses the device in its raw mode but presents a
simple API to the application for reading random binary data. That that respect the library
provides the same interface as the driver, except in application space. However, this allows the
library to also do significant statistical analysis on the data stream before returning it to
the application. That analysis can be used to monitor the statistical properties of the data and
alert the application if any biases are found. The goal of the library is thus to keep the
statistical analysis out of kernel space.

\section{Development Environment Architecture}

Since the development is done using SPARK/Ada, AdaCore's GPS is the primary development
environment. That environment includes the ability to program and debug code on the
STM32F4DISCOVERY board. See Section~\ref{chapt:test-plan} for more information about the test
plan for the system.

For convenience, the development platform is the same as the initial target platform: Ubuntu
Linux 20.04. Driver development takes place inside a Virtual machine to avoid the possibility of
destablizing the development platform while debugging the driver.
